\documentclass{article}
%\documentclass[a4paper,landscape]{article}

\usepackage{graphicx}
\usepackage{tabularx}
\usepackage{adjustbox}
\usepackage{amsmath}
\usepackage{amsfonts}
\usepackage{multirow}

%% Key definitions for text elements. USE THEM
\def\secref#1{Sec.~\ref{#1}}
\def\figref#1{Fig.~\ref{#1}}
\def\tabref#1{Tab.~\ref{#1}}
\def\eqref#1{Eq.~(\ref{#1})}
\def\algref#1{Alg.~\ref{#1}}
\def\appref#1{App.~\ref{#1}}

\newcommand\etal{\emph{et al.}}

\usepackage{rotating}



%\usepackage{geometry}
%\geometry{
%	a4paper,
%%	total={170mm,257mm},
%	left=0mm,
%	top=0mm,
%	right=0mm,
%	bottom=0mm,
%}



\begin{document}
	

	\section{Motivation}
	
	\begin{itemize}
		\item Traditional mapping techniques return a geometric model of the environment that can be used by the robot for computing distances to obstacles or finding paths toward goals.
		\item Manipulation and navigation tasks may require human level knowledge (e.g., objects/places categories, functions, properties, and so on) to be accomplished.
		\item By extracting semantic information from sensory data it's possible to fullfill this requirement.
	\end{itemize}
	
		
	\section{Definition}
		
	Semantic mapping:
		
	\begin{itemize}
		\item is the problem of building a model of the environment that allows to share knowledge between humans and robots.
		\item has the function of providing the robot with a representation of the environment for the completion of its tasks.
		\item is a core component of many robotic applications.
	\end{itemize}
		
	\section{Building a semantic representation of the environment}
	
	The process of semantic mapping consists in integrating sensor measurements in the robot internal representation of the environment.
	
	The output of visual and range sensors is processed to extract meaningful information for recovering the structure and objects/places categories of the scene.
		
	This information is consecutively used to update the current robot belief about the state of the environment.
	
	\section{System Overview}
	
	\section{Perception}
	
	\begin{itemize}
		\item pipelined
		\item geometric only
		\item semantic only
		\item parallel
		\item joint
	\end{itemize}
	
	\section{Recognition}
	
	\begin{itemize}
		\item image classification
		\item object detection
		\item image segmentation
		\item scene labeling
		\item 3d object recognition
		\item 3d object detection
	\end{itemize}
	
	\section{Reconstruction}
	
	\begin{itemize}
		\item metric
		\item topological
	\end{itemize}
	
	\section{Map Management}
	
	\begin{itemize}
		\item offline
		\item frame-by-frame
	\end{itemize}
	
	\section{Matching}
	
	\begin{itemize}
		\item ...
	\end{itemize}

	\section{Symbol Grounding}
	
	\begin{itemize}
		\item context aware
		\item bayesian
		\item local
	\end{itemize}
	
	\section{Action}
	
	\begin{itemize}
		\item ...
	\end{itemize}
	
	\section{Active Vision}
	
	\begin{itemize}
		\item ...
	\end{itemize}
	
	\section{Exploration}
	
	\begin{itemize}
		\item ...
	\end{itemize}
		
	\clearpage
	
	\documentclass{article}

\usepackage{graphicx}
\usepackage{amsmath}
\usepackage{amsfonts}
\usepackage{multirow}

%% Key definitions for text elements. USE THEM
\def\secref#1{Sec.~\ref{#1}}
\def\figref#1{Fig.~\ref{#1}}
\def\tabref#1{Tab.~\ref{#1}}
\def\eqref#1{Eq.~(\ref{#1})}
\def\algref#1{Alg.~\ref{#1}}
\def\appref#1{App.~\ref{#1}}

\newcommand\etal{\emph{et al.}}

%\usepackage{amsopn}


\newcommand{\bJidx}[1]{\ensuremath{\mathbf{J}^{[#1]}}}
\newcommand{\cose}{\mathrm{cose}}
\newcommand{\bzidx}[1]{\mathbf{z}^{[{#1}]}}
\newcommand{\bxidx}[1]{\mathbf{x}^{[{#1}]}}
\newcommand{\bhidx}[1]{\mathbf{h}^{[{#1}]}}
\newcommand{\bOmegaidx}[1]{\mathbf{\Omega}^{[{#1}]}}
\newcommand{\bSigmaidx}[1]{\mathbf{\Sigma}^{[{#1}]}}
\newcommand{\bHidx}[1]{\mathbf{H}^{[{#1}]}}

\newcommand{\bv}{\mathbf{v}}
\newcommand{\bl}{\mathbf{l}}
\newcommand{\bt}{\mathbf{t}}
\newcommand{\bo}{\mathbf{o}}
\newcommand{\bM}{\mathbf{M}}
\newcommand{\bL}{\mathbf{L}}
\newcommand{\bA}{\mathbf{A}}
\newcommand{\bB}{\mathbf{B}}
\newcommand{\bE}{\mathbf{E}}
\newcommand{\bK}{\mathbf{K}}
\newcommand{\bC}{\mathbf{C}}
\newcommand{\bH}{\mathbf{H}}
\newcommand{\bI}{\mathbf{I}}
\newcommand{\bP}{\mathbf{P}}
\newcommand{\bX}{\mathbf{X}}
\newcommand{\bZ}{\mathbf{Z}}
\newcommand{\bR}{\mathbf{R}}
\newcommand{\bS}{\mathbf{S}}
\newcommand{\bU}{\mathbf{U}}
\newcommand{\bV}{\mathbf{V}}
\newcommand{\bT}{\mathbf{T}}
\newcommand{\bpi}{\mathbf{\pi}}
\newcommand{\btl}{\mathbf{tl}}
\newcommand{\bbr}{\mathbf{br}}


\newcommand{\iD}{\mathbf{D}}
\newcommand{\iN}{\mathbf{N}}
\newcommand{\iI}{\mathbf{I}}

\newcommand\norm[1]{\left\lVert#1\right\rVert}

\newcommand{\bzridx}[1]{\ensuremath{\mathbf{z}^{({#1})}}}
\newcommand{\bxridx}[1]{\mathbf{x}^{({#1})}}
\newcommand{\bhridx}[1]{\mathbf{h}^{({#1})}}

\newcommand{\bJ}{\mathbf{J}}
\newcommand{\bZero}{\mathbf{0}}

\newcommand{\cS}{\mathcal{S}}
\newcommand{\cE}{\mathcal{E}}
\newcommand{\cC}{\mathcal{C}}
\newcommand{\cSM}{\mathcal{SM}}
\newcommand{\cR}{\mathcal{R}}
\newcommand{\cM}{\mathcal{M}}
\newcommand{\cP}{\mathcal{P}}
\newcommand{\cL}{\mathcal{L}}
\newcommand{\cD}{\mathcal{D}}
\newcommand{\cZ}{\mathcal{Z}}
\newcommand{\cX}{\mathcal{X}}
\newcommand{\range}[3]{#1_{#2:#3}}


\newcommand{\ba}{\mathbf{a}}
\newcommand{\bb}{\mathbf{b}}
\newcommand{\bc}{\mathbf{c}}
\newcommand{\bd}{\mathbf{d}}
\newcommand{\be}{\mathbf{e}}
\newcommand{\ec}{\mathbf{e}}
\newcommand{\bm}{\mathbf{m}}
\newcommand{\bg}{\mathbf{g}}
\newcommand{\Dim}{\mathrm{Dim}}

\newcommand{\bs}{\mathbf{s}}
\newcommand{\bx}{\mathbf{x}}
\newcommand{\by}{\mathbf{y}}
\newcommand{\br}{\mathbf{r}}
\newcommand{\bz}{\mathbf{z}}
\newcommand{\bu}{\mathbf{u}}
\newcommand{\bn}{\mathbf{n}}
\newcommand{\bh}{\mathbf{h}}
\newcommand{\bff}{\mathbf{f}}
\newcommand{\bp}{\mathbf{p}}
\newcommand{\bDelta}{\mathbf{\Delta}}
\newcommand{\bGamma}{\mathbf{\Gamma}}
\newcommand{\bDeltaalpha}{\mathbf{\Delta \alpha}}
\newcommand{\bDeltar}{\mathbf{\Delta r}}
\newcommand{\bDeltax}{\mathbf{\Delta x}}
\newcommand{\bDeltaX}{\mathbf{\Delta X}}
\newcommand{\bDeltat}{\mathbf{\Delta t}}
\newcommand{\bDeltaR}{\mathbf{\Delta R}}
\newcommand{\tTov}{\mathrm{t2v}}
\newcommand{\vTot}{\mathrm{v2t}}

\newcommand{\bO}{\mathbf{O}}

\newcommand{\defeq}{=}


\newcommand{\bmu}{\mathbf{\mu}}
\newcommand{\bnu}{\mathbf{\nu}}
\newcommand{\bSigma}{\mathbf{\Sigma}}
\newcommand{\bOmega}{\mathbf{\Omega}}
\newcommand{\bLambda}{\mathbf{\Lambda}}

\newcommand{\mat}[1]{#1}
\newcommand{\mbf}[1]{\mathbf{#1}}
\newcommand{\defn}[1]{\emph{#1}}

\newcommand{\mysum}{\sum}
\newcommand{\myprod}{\prod}
\newcommand{\eq}{=}
\newcommand{\pv}{\mathrm{P}}
%\newcommand{\implies}{\Rightarrow}
\newcommand{\Parents}{\mathrm{Parents}}
\newcommand{\rj}{\mathrm{j}}
\newcommand{\proj}{\mathrm{proj}}
\DeclareMathOperator*{\argmax}{argmax}
\DeclareMathOperator*{\argmin}{argmin}
\DeclareMathOperator*{\atantwo}{atantwo}

\newcommand{\mR}{\mathbb{R}}
\newcommand{\mN}{\mathbb{N}}
\newcommand{\mC}{\mathbb{C}}



\begin{document}
			
	\begin{table}
		%\centering
		\begin{tabular}{|c|c|c|c|c|c|c|c|}
			\hline
			 & Matching & Update & Exploration & Active Vision & Filtering & Recognition & Extraction \\
			\hline
			\cite{ulrich2000icra} &  &  &  &  &  & x &  \\ 
			\hline 			 
			\cite{swain1991ijcv} &  &  &  &  &  & x &  \\ 
			\hline 			 
			\cite{torralba2003context} &  &  &  &  &  & x &  \\ 
			\hline 			 
			\cite{oliva2001ijcv} &  &  &  &  &  & x &  \\ 
			\hline 			 
			\cite{lisin2005cvpr} &  &  &  &  &  & x &  \\ 
			\hline 			 
			\cite{viola2004ijcv} &  &  &  &  &  & x &  \\ 
			\hline 			 
			\cite{papageorgiou1998iccv} &  &  &  &  &  & x &  \\ 
			\hline 			 
			\cite{freund1997jcss} &  &  &  &  &  & x &  \\ 
			\hline 			 
			\cite{dalal2005cvpr} &  &  &  &  &  & x &  \\ 
			\hline 			 
			\cite{krizhevsky2012nipsjournal} &  &  &  &  &  & x &  \\ 
			\hline 			 
			\cite{redmon2016cvpr,erhan2014cvpr,liu2016eccv} &  &  &  &  &  & x &  \\ 
			\hline 			 
			\cite{long2015cvpr} &  &  &  &  &  & x &  \\ 
			\hline 			 
			\cite{simonyan2014very,szegedy2015cvpr} &  &  &  &  &  & x &  \\ 
			\hline 			 
			\cite{de2017cvpr} & x & x &  &  &  & x &  \\ 
			\hline 			 
			\cite{stuckler2012iros} & x & x &  &  &  & x &  \\ 
			\hline 			 
			\cite{herbst2014icra} & x & x &  &  &  & x &  \\ 
			\hline 			 
			\cite{vineet2015icra} & x & x &  &  &  & x &  \\ 
			\hline 			 
			\cite{zender2008ras} & x & x &  &  &  & x &  \\ 
			\hline 			 
			\cite{nuchter2008ras} & x & x &  &  &  & x &  \\ 
			\hline 			 
			\cite{xiong2010bmvc} & x & x &  &  &  & x &  \\ 
			\hline 			 
			\cite{blaha2016cvpr} & x & x &  &  &  & x &  \\ 
			\hline 			 
			\cite{kundu2014eccv} & x & x &  &  &  & x &  \\ 
			\hline 			 
			\cite{sengupta2013icra} & x & x &  &  &  & x &  \\ 
			\hline 			 
			\cite{hane2013cvpr} & x & x &  &  &  & x &  \\ 
			\hline 			 
			\cite{yamauchi1997cira,yamauchi1998frontier,wang2011frontier} &  &  & x &  &  &  &  \\ 
			\hline 			 
			\cite{senarathne2013efficient,keidar2012robot} &  &  & x &  &  &  &  \\ 
			\hline 			 
			\cite{oriolo2004icra,freda2005icra} &  &  & x &  &  &  &  \\ 
			\hline 			 
			\cite{lavalle1998rapidly} &  &  & x &  &  &  &  \\ 
			\hline 			 
			\cite{bircher2016icra} & & & x &  &  &  &  \\ 
			\hline 			 
			\cite{el2013improved,franchi2009sensor} &  &  & x &  &  &  &  \\ 
			\hline 			 
		\end{tabular}
		\caption{Data association.}
		\label{tab:assoc}
	\end{table}
	
	\clearpage
	\bibliography{references}
	\bibliographystyle{plain}
	
\end{document}
	
	\clearpage
	
	\bibliography{references}
	\bibliographystyle{apalike}
	
\end{document}